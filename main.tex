\documentclass[11pt]{article}
\usepackage[utf8]{inputenc}
\usepackage[legalpaper, portrait, margin=0.8in]{geometry}
\usepackage{graphicx}
\graphicspath{ {images/} }
\usepackage{float}
\usepackage{hyperref}
\usepackage{todonotes}
\usepackage{ragged2e}
\usepackage{amsmath}
\usepackage{caption} 
\captionsetup{width=1\linewidth}
\usepackage{tabularx}
\usepackage{dirtytalk}
\usepackage{titlesec}

\titleformat*{\section}{\LARGE\bfseries}
\titleformat*{\subsection}{\Large\bfseries}

\usepackage{array,booktabs}
\usepackage[linesnumbered,ruled]{algorithm2e}


\makeatletter

\def\@setref#1#2#3{%
  \ifx#1\relax
   \protect\G@refundefinedtrue
   \nfss@text{\reset@font\bfseries\Large\textcolor{red}{???}}%
   \@latex@warning{Reference `#3' on page \thepage \space
             undefined}%
  \else
   \expandafter#2#1\null
  \fi}

\makeatother


\newcommand{\ttt}[1]{\texttt{#1}}

\setlength\parindent{0pt}
\setlength{\parskip}{1.em}
\title{
    \huge{\textbf{Architecture-Driven \\ Development}} \\
    \includegraphics[scale = 0.26, trim={0 3cm 0 0}, clip]{st-andrews-logo.png} 
}
\author{170002815 $\cdot$ 170007256 $\cdot$ 190004947 $\cdot$ MATRIC NO \vspace{8mm}  \\  Word Count: \vspace{3mm} \\ \\ CS5033 $\cdot$ Practical 2}
\date{22 March 2022}

\begin{document}
\maketitle

% Labels should be formatted in the following way
% - Section:    {sect: SECTION}
% - Subsection: {sect: SECTION: SUBSECTION}
% - Figure:     {fig: FIGURE}
% - Table:      {table: TABLE}

% Subsubsections should avoid numbering:
% \subsubsection*{SUBSUBSECTION}

% Example section references:
% LONG NOTATION:
% Section \ref{sect: SECTION: SUBSECTION}
% SHORT NOTATION:
% \S\ref{sect: SECTION: SUBSECTIOn}

% Example Figure (note images directory path is automatically added):
% \vspace{4mm}
% \begin{figure} [H]
%   \centering
%   \includegraphics[scale = 0.65]{IMAGE.png}
%   \caption{Some caption}
%   \label{fig: LABEL}
% \end{figure}

% Example quote:
% \say{QUOTE}

% Example bold, italic & monospace:
% \textbf{BOLD} \textit{ITALIC} \ttt{MONO}

% Example TODO:
% \todo{NOTE OF WHAT TO DO}

% Example missing figure
% \vspace{4mm}
% \begin{figure} [H]
%   \centering
%   % \includegraphics[scale = 0.65]{IMAGE.png}
%   \missingfigure{WHAT IS MISSING?}
%   \caption{Some caption}
%   \label{fig: LABEL}
% \end{figure}

% Example website reference (place in refs.bib)
% @misc{DETAILED-REFERENCE-NAME,
%     url = {URL},
%     title = {{TITLE, WEBSITE}},
%     note = {{Accessed: 2022-03-DAY}},
% }




\section{Overview}
\label{sect: overview}
This report explores an example architecture for a healthcare system which handles the management of patient information and appointments. The process of creating the architecture is explored in Sections \ref{sect: process} and \ref{sect: requirements}, with the resulting architecture being described in Section \ref{sect: architecture}; Sections \ref{sect: implementation} and \ref{sect: evaluation} provide an implementation plan for and a critical evaluation of the chosen architecture respectively. The report is completed by a conclusion (\S\ref{sect: conclusion}) and set of references.
% A brief overview of the report (not in spec)
\todo{word count}



\section{Architecting Process}
\label{sect: process}
\subsection{Resources}
\label{sect: process: resources}

The primary literature resources used to influence the design, evaluation, and implementation plan of the chosen architecture were Robert C. Martin's Clean Architecture \cite{cleanArch-WholeBook}, and Mark Richards and Neal Ford's Fundamentals of Software Architecture \cite{fundamentalsOfSWA-WholeBook}.

Clean Architecture provides information on architecture from a practical standpoint, including fundamental principles to follow when architecting components and code, as well as a critical analysis of modern software-architecture conventions. The principles relating to the architecture of code which explored in the book are the SOLID design principles, which are summarised as follows.

\todo{summarise}

\begin{itemize}
  \item \textbf{Single-Responsibility Principle (SRP)}:
  \item \textbf{Open-Closed Principle (OCP)}:
  \item \textbf{Liskov Substitution Principle (LSP)}:
  \item \textbf{Interface Segregation Principle (ISP)}:
  \item \textbf{Dependence Inversion Principle (DIP)}:
\end{itemize}

The principles relating to the architecture of components which are explored Clean Architecture are divided into two groups: those which deal with component cohesion, and those which deal with component coupling. The principles relating to component cohesion are as follows.

\begin{itemize}
  \item \textbf{Reuse/Release Equivalence Principle (REP)}:
  \item \textbf{Common Closure Principle (CCP)}:
  \item \textbf{Common Reuse Principle (CRP)}
\end{itemize}

The principles relating to component coupling in Clean Architecture are as follows.

\begin{itemize}
  \item \textbf{Acyclic Dependencies Principle (ADP)}:
  \item \textbf{Stable Dependencies Principle (SDP)}:
  \item \textbf{Stable Abstractions Principle (SAP)}:
\end{itemize}

% A brief description of the process used to derive the architecture (you are not required to document every step)




\section{Requirements Analysis}
\label{sect: requirements}
% A list of functional and non-functional requirements of the system along with their priorities, and any design or business constraints

\subsection{Functional Requirements}
\label{sect: requirements: functional}

-The system shall be able to create and facilitate digital appointments.

-The system shall allow patients to create digital appointments.

-The system shall allow General Practitioners to create digital appointments.

-The system shall be able to securely store sensitive patient data.

-The system shall be able to authenticate patients that interact with the application, by verifying their details against data obtained from the General Practitioners' records.

-The system shall be able to authenticate General Practitioners, General Practice Administrators, Nurses, Ambulance Service Administrators and Health Board Administrators that interact with the application, by verifying their details against data obtained from the National Medical Professionals Database.

-The system shall only allow patients to be referred to the General Practice, following a digital appointment, except in emergencies.

-The system shall allow the patient records to be updated by the General Practitioner.

-The system shall allow the General Practitioner to refer the patient to the Local Health Board if the required service is available locally.

-The system shall only allow the General Practitioner to refer the patient to the National Health Board if the required system is not available locally.

-The system shall allow the patient details to be updated by their General Practitioner, following a digital appointment.

-The system shall allow the patient records to be updated by the General Practitioner, to indicate that a digital appointment has taken place.

-The system shall allow the patient records to be updated by the Health Board Administrator, when the patient goes to A\&E in a Hospital, to indicate that treatment of some kind has taken place.

-The system shall allow Hospital doctors to view patient details and vitals.

-The system shall allow Hospital doctors to add entries on diagnoses made.

-The system shall allow Hospital doctors to add entries on any treatment given.

-The system shall notify the General Practitioners when the Ambulance Service Administrator or Health Board Administrator has updated the patient details.

-The system shall allow the General Practitioners to create digital appointments for patients.

-The system shall allow the General Practitioner to view patients' vitals, such as blood pressure, heart rate, heart-rate variability, oxygen saturation and respiration rate.

-The system shall allow patients to view their vital data that is stored within the system.

-The system shall allow patients to view any changes or new entries made to their vital data, as a result of their digital appointment.

-The system shall allow General Practitioners to access the full record of any patient.

-The system shall allow General Practitioners to order tests for a specific patient that is affiliated with the Practice.

-The system shall allow Nurses to view details of patients' treatments, tests or test results.

-The system shall not allow Nurses to view patients' sensitive information or vitals.

-The system shall allow General Practice Administrators to view appointment details between patients and General Practitioners.

-The system shall allow General Practice Administrators to make or delete appointments, on the behalf of patients.

-The system shall allow General Practice Administrators to obtain a general overview of the Practice's day-to-day activities.

-The system shall allow General Practice Administrators to create statistical reports of the Practice's performance, such as the number of appointments per month, the number of test kits ordered, etc.

-The system shall not allow General Practice Administrators to view any patient information.

-The system shall allow Ambulance Service Administrators to log calls in real-time.

-The system shall allow the Ambulance Service Administrators to view the specific regions where an ambulance is available.

-The system shall allow the Ambulance Service Administrators to dispatch an ambulance that is closest to the region where the call was logged.

-The system shall allow the Ambulance Service Administrators to view patient records and details.

-The system shall allow the patient records to be updated by the Ambulance Service Administrator, to indicate that treatment of some kind has taken place.

-The system should allow Ambulance Service Crew to indicate when each call-out has been addressed.

-The system shall shut down, in the event of a potential cyber threat.




\subsection{Non-Functional Requirements}
\label{sect: requirements: non-functional}

- -The system shall be able to handle all users that attempt to access, update or modify patient details.

-The system shall be able to handle all users that attempt to access or update treatments given.

-The system shall be able to handle all users that attempt to access or update diagnoses made.

-The system shall be able to support 3,000 concurrent users.

-The system shall be able to process all identifiable data under the Data Protection Act,
1998.

-The system shall be able to obtain real-time patient vitals in 0.1 seconds.

-The system shall be able to load all operations within 1 second.

-The system shall be able to process all requests within 2 seconds.

-The system shall be available between 24 hours a day, 7 days a week, with the exception of a maintenance window.

-The system should have centralised logs that can maintain all services, instances and
possible errors in a single location.

-The system shall be accessible via mobile devices and personal computers.

-The system shall be compatible with the latest product-release OS versions.

-The system shall be able to interface with Binah.AI, in order to enable video-calling and obtain real-time patient vitals.






\subsection{Organisational Requirements}
\label{sect: requirements: organisational}

-Users of this system shall identify themselves using their National Medical Professionals' Database credentials.

-The system shall be down for maintenance, on the last working day of each calendar month, for 30 minutes.

-The system shall be evaluated for response times on the last working day of each calendar month.

-The system shall be able to prevent cross-scripting attacks.

-The system will not store unencrypted sensitive data.

-Database security must meet the Health Insurance Portability and Accountability Act (HIPAA) requirements.

-All patient data will be retained in system archives for up to 5 years.



\subsection{External Requirements}
\label{sect: requirements: external}

-The system should be able to support an annual growth of 10 General Practices that can utilise this platform.

- The system should be able to support a 100\% growth in user concurrency, and still meet
all defined functional and non-functional requirements.

- The system should utilise cloud-based solutions that facilitate scalability.




\subsection{Assumptions}
\label{sect: requirements: assumptions}

For the purposes of this practical, there were a number of assumptions that were made, in order to streamline the process of designing a system software for General Practices and to consolidate the expectations of this platform.

These assumptions and their justifications, are listed as follows.


\begin{enumerate}

  \item \textbf{We assume that a Health Board Administrator is responsible for redirecting a referred patient to the nearest Hospital that offers their required service. }
        For example, a student based in St. Andrews would have their patient details and records stored within the Fife Health Board. This is the Health Board that the local General Practice (example- Blackfriars or Pipeland) would refer the patient to.
        A Health Board could encompass multiple Hospitals - for example, the Fife Health Board could include Adamson Hospital (Cupar), Victoria Hospital (Kirkcaldy), St Andrews Community Hospital, etc. Thus, when a patient is “referred” to a local Health Board, we assume that the General Practice does not refer the patient to a specific Hospital.
        Hence, we assume that a Health Board Administrator is responsible for selecting the specific Hospital that the patient will be referred to, and as such, will pass the patient's details (vitals, treatment records, diagnoses, personal information, etc) onto this specific hospital. This will ensure that the patient's details are communicated on a need-to-know basis, thereby reducing the chances of sensitive information being released to inappropriate individuals.

  \item \textbf{We assume that the Ambulance Service Administrator is able to dispatch ambulances that are nearest to the individual calling for assistance. }
        This indicates that the Ambulance Service Administrator retains a degree of autonomy when dispatching ambulances, in order to ensure that assistance is provided as quickly as possible.
        An example of a scenario where this would be useful would be a call requiring assistance to be sent to St Andrews. If there are two ambulances in Kirkcaldy and Cupar, we would want the Ambulance Service Administrator to be able to send the ambulance in Cupar, as it is closer.
        Hence, we do not expect the system to be automatically dispatching ambulances, as the calls come in.
        However, we do expect the system to be able to provide a recommendation of which ambulance is closest (computed using third-party navigation software), which would help the Ambulance Service Administrator dispatch assistance.


  \item \textbf{We assume different members of the General Practice/Health Board/Ambulance Crew have different access permissions, regarding patient data. }
        Primarily, we assume that General Practitioners have access to all patient information. This includes the patient vitals (heart-rate, heart-rate variability, oxygen saturation, respiration rate, parasympathetic activity, etc), sensitive patient information (date of birth, gender, height, weight, etc), patient diagnoses and treatment.
        We assume that Nurses only have access to patient diagnoses and treatment.
        We assume that General Practice Administrators have access to no patient data. We assume that these individuals can only view anonymised appointment data.
        We assume that Health Board Administrators have access to no patient data, except the information that the General Practitioner has sent over, regarding the services required by the patient.
        We assume that the Hospital doctors have read-only access to patients' vitals and sensitive data, but can update patient diagnoses and treatment.
        We assume that the Ambulance Service Administrators have no access to patient data, except the data communicated by the patient during the ambulance call.
        We assume that the Ambulance Service Crew have read-only access to patient vitals and sensitive patient information, but can update patient diagnoses and treatment.


  \item \textbf{We assume that the patient does need to make a digital appointment when being seen by a Nurse/General Practitioner in person.}
        We assume that the patient must make an appointment by phoning the General Practice, in order to be seen by a Nurse or General Practitioner in real life. This allows us to model real-world behaviour as closely as possible, as we would expect that General Practices require an appointment to be made, in advance. This holds true whether it is a digital appointment or an in-person appointment.
        We also assume that the steps following an in-person appointment are the same as those that follow a digital appointment. Specifically, the patient's vitals are taken and added to the database, treatment is offered/the patient is referred and test kits are ordered.


  \item \textbf{We assume that all medical personnel records are registered in a comprehensive database - the National Medical Professionals' Database.}
        We assume that the credentials utilised to authenticate users are obtained from a centralised database that contains the information of all registered medical professionals within the country. This allows the system architecture to ensure that only authorised medical personnel are allowed to access patient records and that any externally created user accounts (such as system administrators) are automatically unauthorised to view sensitive information.

\end{enumerate}

% A list of functional and non-functional requirements of the system along with their priorities, and any design or business constraints
% AND
% Any assumptions made regarding the target system




\section{Architecture}
\label{sect: architecture}
\subsection{Modelling Language}
\label{sect: architecture: modelling language}

A relatively informal modelling language is used to describe the final architecture produced for this report, with the C4 Model being employed in combination with a notation similar to UML. This choice was made as all team members were relatively familiar with the concepts used in both practices, focussing the project's time to designing and evaluating the architecture, rather than learning modelling languages.



\subsubsection*{C4 Model}
The C4 Model primarily consists of four different architectural viewpoints, described in the form of diagrams, which represent an architecture as a hierarchical set of abstractions \cite{c4Model}. The viewpoints are designed to reflect how software architects and developers think about and build software, and are summarised as follows.

\begin{itemize}
  \item \textbf{System Context Diagram}: A viewpoint which describes a system's architecture as a whole, allowing viewers to see the \say{big picture}. The diagram has a focus on the users of the system, rather than specific details, such as which technologies are used. Section \ref{sect: architecture: overview} explores the system context diagram for the chosen architecture in this report.

  \item \textbf{Container Diagram}: A viewpoint which shows the details of a single subsystem within a system context diagram. The diagram is composed of containers, which represent an application or datastore (e.g. a web application, filesystem, database, etc.), and high-level references to other subsystems. Sections \ref{sect: architecture: appt svc}, \ref{sect: architecture: ambulance svc}, \ref{sect: architecture: gp notification svc}, and \ref{sect: architecture: patient information svc} display container diagrams for the subsystems which they describe.

  \item \textbf{Component Diagram}: A viewpoint which describes a single container within a container diagram, with respect to the major building blocks (i.e. components) that make up the container. Component diagrams contain some specific details, such as the technology used to implement or communicated between services, but still provide a relatively high-level overview of a container. Only a limited number of component diagrams are used to describe the proposed architecture for the sake of brevity; Sections \ref{sect: architecture: patient information svc} and \ref{sect: architecture: gp notification svc} display a component diagram for the parts of the architecture which they describe.

  \item \textbf{Code Diagram}: A viewpoint which describes a single component within a component diagram as it is implemented within code, including the classes and interface involved, and their relationships. Code diagrams are not explored in this report, as they reference relatively low-level implementation details, and could be automatically generated from code written during the system's development. No code diagrams are displayed in this report, as they are typically auto-generated from a codebase \cite{c4Model}.
\end{itemize}



The C4 Model also includes three supplementary diagrams: the system landscape diagram, the dynamic diagram, and the deployment diagram. However, none of these are reference within this report, as the four core diagrams are sufficient to capture the chosen architecture \cite{c4Model}.


\subsection{Notation}
A UML-like notation is used within the C4 Model diagrams, as group members were relatively familiar with the technology. The notation is a hybrid between UML \cite{umlSpec} and the suggested C4 Model notation \cite{c4Model} to keep the diagrams understandable and focussed. It is also worth noting that the diagrams are purely structural, they do not indicate how elements interact as this is described elsewhere only that the elements are communicating with each other.





\subsection{Overview}
\label{sect: architecture: overview}

The architecture derived for the scenario listed in Appendix \ref{appendix: scenario} is fundamentally a service-oriented architecture (SOA), as the key functionality of the system is provided by multiple independent, distributed services. The core services used and their roles within the architecture are as follows.

\begin{itemize}
  \item \textbf{Appointments Service}: Handles create, read, update, and delete (CRUD) operations relating to appointments, employing business rules in combination with authentication to authorise information appropriately. This service is described further in Section \ref{sect: architecture: appt svc}.

  \item \textbf{Patient Information Service}: Handles CRUD operations relating to patient information, authorising information similarly to the Appointments Service. This service is described further in Section \ref{sect: architecture: patient information svc}.

  \item \textbf{Ambulance Service}: Affords the logging of calls and the dispatching of ambulances. This service is described further in Section \ref{sect: architecture: ambulance svc}.

  \item \textbf{GP Notification Service}: Provides the ability to asynchronously pass notifications to GPs. This service is described further in Section \ref{sect: architecture: gp notification svc}.
\end{itemize}


In order to enable user interaction with services, a client-server-like architectural style is employed for user-service interactions; Section \ref{sect: architecture: clients} discusses this further.


Figure \ref{fig: arch overview / system context diagram} displays the system context diagram (see \S\ref{sect: architecture: modelling language}), where the set of core services, users, and interactions can be seen.


\vspace{4mm}
\begin{figure} [H]
  \centering
  \includegraphics[scale = 1]{context.png}
  \caption{The system context diagram for the proposed architecture, displaying the service-oriented approach used as the fundamental architectural style for the system.}
  \label{fig: arch overview / system context diagram}
\end{figure}



\subsubsection*{Service-Oriented Architecture}

As the target system's domains can be partitioned with relative ease, an SOA is a suitable choice for the overarching architectural style \cite{fundamentalsOfSWA-SOA}. Although there are other applicable architectural styles for the target system, the following benefits of an SOA were the primary driving factors in the decision to prioritise it over other architectural styles.


\begin{itemize}
  \item Services can be developed independently, reducing time-to-market, as independent teams can develop each service, and improving business agility, as each service's business rules can be adapted independently, provided the boundaries within the architecture are enforced effectively \cite{ibmSOA}. This is primarily an advantage over more rigid architectures, such as those employing a monolithic style.

  \item Services can be deployed independently, facilitating the maintainability, short-term scalability (elasticity), and long-term scalability of the target system. This is primarily an advantage over architectures which are designed to have a one-to-one mapping from system instance to deployment machine, such as an architecture employing a layered monolithic style.

  \item Services are more independent with respect to failure than other architectural styles. This means that if one service fails, the failure of other services is not guaranteed, which is particularly beneficial for the target system, as the domains are mostly disconnected. For example, if the GP Notification Service fails, the other services in the system are not guaranteed to fail, meaning that the system will still provide the majority of its functionality to users. Overall, this improves the system's fault-tolerance.

  \item Similarly to failure, the services are more independent with respect to performance than other architectural styles. For example, if the GP Notification Service handles requests slowly due to a bug in its implementation, the Ambulance Service would be unaffected in terms of performance, provided it is deployed on a different machine; this is particularly beneficial for the target system, as some services, such as the Patient Information Service, require a certain level of performance to function as required.

  \item Due to the network-based nature of an SOA, services and clients are typically designed to be more robust to failure, improving the overall reliability of the system.

  \item Integration with external systems, such as Hospitals (see requirement 8 in Section \ref{sect: requirements: functional}), is facilitated, as the target system will be deployed on a network due to the nature of an SOA. Similarly, integration with legacy systems which might need to be included in the newly developed system is facilitated \cite{ibmSOA}.

  \item The enforcement of architectural boundaries of the target system's domain is improved, as each service's boundary can be mapped directly to each domain's architectural boundary \cite{fundamentalsOfSWA-SOA}.

  \item The domain-driven separation of services, in combination with their low levels of interaction, mitigate the negative effects of an SOA, including selecting an appropriate granularity and handling service choreography/orchestration \cite{fundamentalsOfSWA-SOA}.
\end{itemize}






\newpage
\subsection{Appointments Service}
\label{sect: architecture: appt svc}

The Appointments Service handles read and write requests relating to patient appointments. The service is designed with a relatively typical architecture for a read-heavy workloads, consisting of a service which contains the business logic for the Appointments Service, a datastore, and a cache. Figure \ref{fig: appt svc container diagram} displays the container diagram for the Appointments Service.

\vspace{4mm}
\begin{figure} [H]
  \centering
  \includegraphics[scale = 1]{appointment-container.png}
  \caption{The Appointments Service described using a container diagram.}
  \label{fig: appt svc container diagram}
\end{figure}



\subsubsection*{Datastore}
The Appointments Service's datastore is used to persist data relating to appointments, improving the service's reliability. The datastore is integrated within the Appointments Service's architecture in such a way that the datastore component is dependent on the service component within the Appointments Service, rather than the other way round. This integration style employs the Stable Abstractions Principle (see \S\ref{sect: process: resources}), as the dependency is pointing from the less stable datastore to the more stable business rules contained within the service component.

The application of the SAP also follows the advice given in Martin's Clean Code regarding how datastores should be integrated within architectures \cite{cleanArch-databasesAreDetails}, meaning that the choice of data storage technology is not confined by the service's implementation, allowing for solutions ranging from a simple file system to a complex RDBMS to be used, improving both the maintainability and scalability of the Appointments Service.



\subsubsection*{Cache}
To improve the performance of the Appointments Service with relatively low overhead, an in-memory cache is used to respond to some requests without querying the datastore. In order to do so, the cache would store requests and their respective results as they are returned from the service after they had been retrieved from the datastore for the first time.

Two key considerations when introducing a cache into a system are how consistency and coherency will be handled during its operation. Handling these concerns, particularly if the cache or service to which the cache is applied to is replicated, can quickly become a complex problem which introduces a significant amount of overhead \cite{dataIntensiveApps-ConsistencyAndConsensus}. Fortunately, as the Appointments Service's workload is read-heavy, the trade-off of increased overhead is likely worth it when considering the potential performance gains introduced by the cache. Other overhead introduced by the cache, which includes the storing of entries, checking of entries, and handling evictions, is considered to be marginal when compared against the potential performance gains provided by the cache.

As it is assumed that there are many users, as the system is designed for the NHS, it is likely that evicting items from the cache on the commonly employed lease-recently-used (LRU) basis is inefficient. This is because the cache will quickly fill up with requests relating to individual users which cannot be reused for other users (e.g. fetching a user's set of appointments), potentially evicting cached information which can be reused (e.g. the grand total of number of appointments for all users). While an alternative cache eviction policy, such as a least-frequently used (LFU) or re-reference interval prediction (RRIP) policy, it considered as an implementation detail to determine this, and is therefore excluded from the scope of the architecture.






\newpage
\subsection{Patient Information Service}
\label{sect: architecture: patient information svc}

The Patient Information Service handles all create, read, update, and delete (CRUD) operations relating to patient information, including personal information, such as patient addresses, as well as medical information, such as blood pressure readings. The service employs the use of role-based authentication to provide different users access to different data, as information handled by the service is considered highly sensitive \cite{EUHealthDataProtection} and different users need different access permissions as per the system requirements (see \S\ref{sect: requirements: functional}).


Figure \ref{fig: patient information container diagram} displays a container diagram for the Patient Information Service.

\vspace{4mm}
\begin{figure} [H]
  \centering
  \includegraphics[scale = 1]{patient-container.png}
  \caption{The Appointments Service described using a container diagram.}
  \label{fig: patient information container diagram}
\end{figure}


One of the key components of the Patient Information Service is the Service component. Figure \ref{fig: patient information component diagram} displays a component diagram for this component, with the arrows representing the direction of dependency between the constituent parts.


\vspace{4mm}
\begin{figure} [H]
  \centering
  \includegraphics[scale = 0.9]{pateint-component.png}
  \caption{The Appointments Service's Service component described using a component diagram.}
  \label{fig: patient information component diagram}
\end{figure}

It can be seen from the dependencies shown in Figure \ref{fig: patient information component diagram} that the Stable Abstractions Principle (see \S\ref{sect: process: resources}) is applied, such that all dependencies are pointing towards an abstract concept, such as an interface or the implementation of the business rules in the Service subcomponent (marked with an asterisk in the diagram).




\subsubsection*{Datastore \& Cache}
As the data handled by the Patient Information Service and the Appointments Service (see \S\ref{sect: architecture: appt svc}) are both related to individual patients and are associated with read-heavy workloads, the datastore and cache used within the Patient Information Service are similar to those used in the Appointments Service (see \S\ref{sect: architecture: appt svc}), with respects to both operation and reasons for introducing them as part of the architecture (i.e. reliability and performance).




\subsubsection*{Multilevel Priority Queue}
One way in which the Patient Information Service differs from the Appointments Service is the requirement to handle bursty write behaviour (i.e. write behaviour that experiences spikes in workload) when patient information is generated by GPs. As this behaviour contrasts significantly to typical read-heavy workloads of the Patient Information Service, an additional architectural feature of a multilevel priority queue is introduced to handle all write requests sent to the service.

A multilevel priority queue is a task queue which consists of multiple queues, each of which associated with a different priority level. Inputted tasks are enqueued onto a queue with respect to their given priority level, with the processing element(s) (e.g. the Service component in the proposed architecture) taking tasks from the data structure based on descending priority, meaning that more important tasks are completed before less important ones \cite{osConceptsMultilevelQueue}.

A multilevel priority queue is employed within proposed architecture to support the unique bursty write workloads without interrupting the standard non-bursty write workloads. Bursty write workloads, such as the updating of patient information, gathered during testing procedures, are prioritised lower than non-bursty workloads, such as standard updates to patient information (e.g. a change in address), with the service component handling tasks in an order with respect to their priority. The queue nature of the multilevel priority queue also enables for write requests to be delayed as to avoid disrupting and invalidating read workloads, which are likely more important.

Although multilevel priority queues are commonly implemented as multilevel feedback queues, where tasks are moved between queues depending on the amount of time they have been worked on for \cite{osConceptsMultilevelFeedbackQueue}, such an implementation is unsuitable for the target system's architecture, as it is likely unfair to change the priority of each task once it has been assigned.





\subsection{GP Notification Service}
\label{sect: architecture: gp notification svc}
The GP Notification Service is a comparatively simple service which provides the ability to asynchronously pass notifications to GPs through the use of a publish-subscribe messaging system. This service was introduced in the architecture for the following primary reasons.

\begin{enumerate}
  \item The GP Notification Service allows for the GP client device to be decoupled from the rest of the infrastructure which sends notifications to GPs, such as the hospital.

  \item The addition of future features which require GP notifications are greatly facilitated, as the new features would simply need to interact with the GP Notification Service, with no other architectural changes being required.

  \item Future scaling of the system is better supported with a publish-subscribe architecture in comparison to direct communication between services sending notifications and GP interface devices.

  \item The GP Notification Service can buffer notifications sent to GPs, enabling asynchronous communication, which in turn can improve the reliability of the system. For example, if a GPs device briefly goes offline, notifications are not necessarily lost as the GP Notification Service can buffer them while the device is offline; however, if services were to communicate directly with GP devices, then notifications would have likely been lost in this scenario.
\end{enumerate}

Figure \ref{fig: gp notifcation service container diagram} displays a container diagram for the GP Notification Service.

\vspace{4mm}
\begin{figure} [H]
  \centering
  \includegraphics[scale = 1]{gp-container.png}
  \caption{The GP Notification Service described using a container diagram.}
  \label{fig: gp notifcation service container diagram}
\end{figure}



One of the key components within the Appointments Service is the Service component; Figure \ref{fig: patient information component diagram} displays the component diagram for this component.

Similarly to the Service component within the Appointments Service (see \S\ref{sect: architecture: appt svc}), the Stable Abstractions Principle and Common Closure Principle (see \S\ref{sect: process: resources}) are applied, such that dependencies point towards more stable components.

\vspace{4mm}
\begin{figure} [H]
  \centering
  \includegraphics[scale = 0.9]{gp-component.png}
  \caption{The GP Notifications's Service component described using a component diagram.}
  \label{fig: gp notifcation service component diagram}
\end{figure}



\subsubsection*{Publish-Subscribe Pattern}
The GP Notification Service applies an event-driven architectural style in the form of a publish-subscribe (pub-sub) system which buffers notifications sent from other services.

A pull configuration is used, where the GP client device receives notifications by polling the GP Notification Service, rather than a push configuration, where the GP Notification Service would actively communicate with the GP client device to deliver notifications, for the following reasons.

\begin{enumerate}
  \item The GP devices are external to the system, meaning that initialising communication from the GP Notification Service to each GP device is complicated to achieve in comparison to the other way around \cite{googleCloudPubSub}.

  \item The rate of communication (e.g. polling frequency) can be dynamically adjusted based on the GP device's current network connection. This is more beneficial than the GP Notification Service having this control, as the GP device's network connection is likely to fluctuate more than the GP Notification Service's connection given that it is a user device.

  \item The polling interval can be adjusted for each individual GP by the GP's device, allowing for a trade-off in network usage and the amount of delay caused by the gap between the notification being received and polled. For example, if the GP's device has a poor internet connection or is battery powered, the rate of polling could be decreased to improve network efficiency or reduce power consumption.

  \item The notifications being sent to GPs are assumed to not be time-critical, meaning that the delay of messages caused by a pull configuration does not detract from the functionality of the system.
\end{enumerate}





\newpage
\subsection{Ambulance Service}
\label{sect: architecture: ambulance svc}

The Ambulance Service handles all ambulance-related concerns for the target system, including the logging of calls, the dispatching of ambulances, and the indication of callouts being dealt with. The ambulance-related requirements of the target system's domain appears relatively disconnected from other requirements, providing good reason for this service to be a part of the chosen architecture for this report.

\vspace{4mm}
\begin{figure} [H]
  \centering
  \includegraphics[scale = 0.65]{ambulance-container.png}
  % \missingfigure{Ambulance Svc Container Diagram}
  \caption{The Ambulance Service described using a component diagram.}
  \label{fig: ambulance svc container diagram}
\end{figure}



\subsubsection*{Publish-Subscribe Pattern}
Similarly to the GP Notification Service (see \S\ref{sect: architecture: gp notification svc}), the Ambulance Service employs an event-driven architectural style in the form of a publish-subscribe (pub-sub) messaging system. The pub-sub system is used primarily for performance purposes, and therefore uses a push configuration for message delivery, contrasting to the pull configuration used for the GP Notification Service.

A push configuration is more performant in comparison to a pull configuration with respect to message delivery latency, because the delay between the message arriving at the Ambulance Service and being sent to the client device is marginal, in comparison to a pull configuration where the delay is on average half of the polling interval \cite{googleCloudPubSub}. Although a push configuration adds complexity from communicating with client devices outside the system's internal network, and reduces flexibility for the power and network consumption for client devices, these trade-offs are accepted in favour of the reduced message delivery delay, as this characteristic is critical to the Ambulance Service's functionality.








\subsection{Clients}
\label{sect: architecture: clients}


The clients within the architecture offer the functionality of the services in the form of a user interface (UI). All clients share a similar architecture, which can be seen in Figures \ref{fig: appt svc container diagram}, \ref{fig: patient information container diagram}, \ref{fig: gp notifcation service container diagram}, and \ref{fig: ambulance svc container diagram}, and provides the support for a variety of different device types, as well as the potential for functionality, such as form validation.


The overarching architectural style applied for communication between clients and services is the client-server style, where the clients, such as patient mobile devices, communicate with internal services, such as the Patient Information Service, through the use of standard web technologies (e.g. HTTP/HTTPS) to offload computation and data-storage responsibilities.

In order to provide seamless, flexible, reliable, and secure integration with the services, the clients exhibit two primary architectural patterns within the client-server style: gateways and backends for frontends.


\subsubsection*{Gateway Pattern}

The gateway architectural pattern employs an intermediate service to mediate requests which pass from client devices to the services within a system, rather than having client devices communicate directly with the services.

The gateway architectural pattern provides the following advantages.

\begin{itemize}
  \item Cross-cutting concerns, such as authentication, and data transformations, can be handled in a uniform manner for all client devices, simplifying the implementation of each user interface \cite{micosoftGatewayPattern}.

  \item The communication protocol used by requests can be changed before being forwarded to services, allowing for client devices to communicate using web-based protocols (e.g. HTTP/HTTPS) with services that are implemented to communicate using non-web-based protocols, such as QUIC \cite{QUICProtocol}\cite{micosoftGatewayPattern}.

  \item As all requests are passed through a single location before entering the internal system where the services lie, the gateway pattern facilitates the application of performance and security measures, such as load balancing and request filtering.

  \item Requests to different services can be coalesced to reduce data ingress and egress for client devices, improving performance. For example, if a patient's mobile device was loading the patient's information and appointments, the two separate requests to the Patient Information Service and Appointments Service could be combined into a single request to the respective gateway node, reducing the required network communication overhead \cite{micosoftGatewayPattern}.
\end{itemize}

The primary disadvantages of the gateway architectural pattern is the increased architectural complexity and the potential performance concerns surrounding the gateway services becoming a bottleneck or single point of failure. As  the gateways used in the chosen architecture in this report have potential to be stateless, they can be replicated to mitigate the issues of performance and reliability; however, the issue of increased architectural complexity cannot be mitigated. Overall, the advantages of the using gateways for the target system's architecture outweigh the disadvantages.




\subsubsection*{Backend For Frontend Pattern}

The backend for frontend (BFF) architectural pattern is the application of one intermediary service (the backends) per client-side user interface (the frontends).

While similar to the gateway pattern, and often considered synonymous \cite{bffsCouldBeConsideredGateways}, the BFF pattern is applied within the chosen architecture for this report for different reasons to the gateway pattern, which are as follows.

\begin{itemize}
  \item The development of frontend applications using the humble object pattern  is greatly facilitated, as the business logic for the applications are contained within the backend services instead. The humble object pattern is the separation of logic from frontend components and increases the testability of such components; the application of the humble object pattern is generally considered to be good software-engineering practice \cite{cleanArch-HumbleObjectPattern}.

  \item Each BFF backend can be tailored specifically to its respective frontend, allowing for platform-specific intricacies to be handled cleanly, in turn facilitating development, and improving performance and maintainability. This is in contrast to the gateway pattern which uses a single backend for all frontends.
\end{itemize}

The trade-offs of using both the BFF and gateway pattern in the target system's architecture are discussed in Section \ref{sect: evaluation: current considerations} of this report.









\subsection{Authentication}
\label{sect: architecture: authentication}

Authentication is a critical part of the system's functionality, as all services require some form of authentication to function in-line with the system's specification. For example, functional requirement 2 in Section \ref{sect: requirements: functional} can be achieved through the use of authentication.

In order to include authentication within the system's architecture, the OAuth 2 framework \cite{OAuth2} was chosen for the following reasons.

\begin{itemize}
  \item It is a proven authentication framework, as it is used by renowned companies such as Facebook, GitHub, and DigitalOcean \cite{digitalOceanOAuth2}
  \item It enables single-sign on for the target system, which is beneficial from a user standpoint \cite{oauth2SSO}.
  \item The protocol used to authenticate users is simple in comparison to other authorisation techniques.
  \item It provides simple integration with third-party services \cite{digitalOceanOAuth2}. This is beneficial for the target system, as external systems, such as those used within hospitals, may require authorisation for their required access to the system.
  \item The primary architectural elements required for the framework can be \say{bolted} onto the rest of the architecture, allowing for the component coupling dependencies to be followed (see \S\ref{sect: process: resources}), in turn improving the flexibility and maintainability of the architecture as a whole.
\end{itemize}
% The architecture specification in terms of appropriate views, including justification of chosen notation(s), styles adopted and design decisions,




\section{Implementation}
\label{sect: implementation}
\todo{deploy on different nodes to ensure failures are indep. If the target user-base was small, it might be worth using a single node for cost, but the size of the NHS user-base is not small}

\todo{see Clean Arch - says start off with single service with clear arch boundaries, then split into multiple services as system grows}
% A brief analysis of the architecture and the implementation plan, including how architectural styles and constraints would be maintained in the implementation





\section{Evaluation}
\label{sect: evaluation}
\todo{we were thinking about two separate info svcs, so maybe argue the case as to why / why not}

\todo{SOA is not \textbf{that} important in an arch, but instead we should focus on boundaries (see Clean Architectrue book ch 27 \cite{cleanArch-Services})}

\todo{experiences of group members?}

\subsubsection*{Data Loss in Event-Driven Styles}
- Ambulance \& GP service can lose info (responsibility down to devs to req-reply)

- Patient Info service priority queues do not back up info
- Tricky to fix, but maybe just better to ignore, as information such as blood pressure can quickly be regathered


\subsubsection*{BFF \& Gateway Patterns}
- using both rather than just one might be considered unusual
- reuse functionality through gateway (e.g. security filters)
- some can't be (e.g. req coalescing)
- Combining both is beneficial
% A brief analysis of the architecture and the implementation plan, including how architectural styles and constraints would be maintained in the implementation
% AND
% A critical evaluation of your solution particularly with respect to important quality attributes




\section{Conclusion}
\label{sect: conclusion}

This report provides an overview of an architecture for the system described in Appendix \ref{appendix: scenario}, along with a detailed breakdown of architecting process and system requirements, an implementation plan, and a critical evaluation of the proposed architecture.

The architecting process described in Section \ref{sect: process} is detailed, and the requirements analysis completed in Section \ref{sect: requirements} is thorough, and considers situations which could be considered ambiguous in the original specification.

The proposed architecture itself employs a wide variety of both common and unusual architectural features, patterns, and styles to shape the proposed architecture towards the needs of the target system, including the application of the client-server pattern (see \S\ref{sect: architecture: clients}) and a multilevel priority queue (see \S\ref{sect: architecture: patient information svc}).

The evaluation completed for the architecture and the proposed implementation plan for the architecture in Section \ref{sect: implementation} are both detailed and demonstrate substantial knowledge of the software architecture paradigm.

Overall, this practical has been completed to a high standard and fulfils the base specification, as well as completing several minor extensions.


% A brief conclusion of the report (not in spec)

Overall, this report provides an overview of an architecture for the system described in Appendix \ref{appendix: scenario}, along with a detailed breakdown of the system's requirements, an implementation plan, and a critical evaluation of the proposed architecture.

A range of functional and non-functional requirements were derived from the

A wide variety of both common and unusual architectural features, patterns, and styles have been applied to shape the proposed architecture towards the needs of the target system, including the use of the simple client-server pattern (see \S\ref{sect: architecture: clients}) and a multilevel priority queue (see \S\ref{sect: architecture: patient information svc}).



\appendix

\RaggedRight
\bibliographystyle{IEEEtran}
\bibliography{refs.bib}

\section{Scenario}
\label{appendix: scenario}

You are required to develop an architecture for a patient digital appointments and
management system.

This system should act as a central point for creating and delivering patient
appointments as well as storing patient data. Digital appointments should be created
via this system, allowing a GP to ascertain and monitor vital patient data. The patient's
own data should also be stored in the system allowing for a full 360 degree view of
patient information. This system is being built in order to alleviate pressure on the health
service.

Each person resident in Scotland is registered with a General Practitioner (GP)
practice. The health care service will maintain a record of each person's name, a unique
health care id, date of birth, contact details, information on next of kin, the GP practice
they are registered with and medical history. Patient records contain highly sensitive
information. A patient may be seen via digital appointments or at the practice by a nurse
(for minor injuries or tests) or a GP. A patient can only access the GP practice if they
are referred to after a digital appointment, except in case of emergencies when they
can either call an ambulance or go directly to the accident and emergency (A\&E)
department of a hospital. An ambulance crew will either treat a patient on site or take
them to a hospital. A GP will either treat a patient themselves or refer the patient to a
hospital run by the local health board, except when the required service is not available
locally. Each appointment or treatment will result in an entry in the person's medical
history, which should be accessible across different health boards and services. When
a patient has been treated at a hospital or by an ambulance crew, a notification should
be sent to their GP to flag up the new entry.

The GP should be able to generate digital appointments for patients. These
appointments should allow the GP (via the patient's personal device) to obtain vital
information about the patient – namely their blood pressure, heart rate, heart rate
variability, oxygen saturation and respiration rate. This data should be generated in
real-time and stored securely in the patient information system.
The system should provide different functionalities depending on the category of user.
A patient must be able to use the service to make appointments at their GP practice.
They should also be able to see all the vital data generated by all their digital
appointments in the system. GPs should be able to access the full record of any patient
without delay, initiate appointments and add entries to the patient record. They should
also be able to order one or more tests for the patient within the practice or refer the
patient to a hospital, either within or outside the local health board. Nurses can see a
limited part of the patient record and add entries relating to treatment, tests or test
results. Practice administrators can make or cancel appointments for patients and
produce statistical reports on the performance of the practice without accessing details
of any patient.


An ambulance service administrator must be able to log calls to the service and
dispatch ambulances to patients who require them. This aspect of the service is highly
time sensitive. They can also update patients' records according to the service
delivered. Ambulance crews indicate when each callout has been dealt with.
Hospital doctors can view all the details of any patient they see and add entries on
diagnoses made and treatment given.

A well-designed and implemented system should also support the following features:
\begin{itemize}
  \item An intuitive UI appropriate to the user category,
  \item Access authentication for different categories of users and restriction of
        available information and functionality accordingly,
  \item Support for multiple types of devices,
  \item Concurrent access,
  \item Support for several different views and analyses over the data,
  \item Validation of input data where applicable,
  \item Generation of digital appointments and using patients' devices functionalities to
        extract the necessary patient data, and
  \item Deal with potential uncertainties when patients' devices cannot extract this data.
\end{itemize}


\end{document}
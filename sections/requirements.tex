% A list of functional and non-functional requirements of the system along with their priorities, and any design or business constraints

During the initial design process (see \S\ref{sect: process}), a set of requirements were derived from the system specification listed in Appendix \ref{appendix: scenario}. The functional requirements from this set are displayed in Section \ref{sect: requirements: functional} and the non-functional requirements are displayed in Section \ref{sect: requirements: non-functional}.



\subsection{Functional Requirements}
\label{sect: requirements: functional}

\begin{enumerate}
  \item The system shall be able to allow patients and General Practitioners to manage, create and facilitate digital appointments.
  \item The system shall be able to securely store sensitive patient data.
  \item The system shall be able to authenticate patients that interact with the application, by verifying their details against data obtained from the General Practitioners' records.
  \item The system shall be able to authenticate General Practitioners, General Practice Administrators, Nurses, Ambulance Service Administrators and Health Board Administrators that interact with the application, by verifying their details against data obtained from the National Medical Professionals Database.
  \item The system shall allow the patient records to be viewed and updated by the General Practitioner.
  \item The system shall allow the General Practitioner to refer the patient to the Local Health Board if the required service is available locally, or to the National Health Board if the required system is not available locally.
  \item The system shall allow the patient records to be updated by the Health Board Administrator, when the patient goes to A\&E in a Hospital, to indicate that treatment of some kind has taken place.
  \item The system shall allow Hospital doctors to view patient details and vitals, and add entries on diagnoses made and treatment given.
  \item The system shall notify the General Practitioners when the Ambulance Service Administrator or Health Board Administrator has updated the patient details.
  \item The system shall allow patients to view their vital data, and any changes made as a result of their digital appointment.
  \item The system shall allow General Practitioners to order tests for a specific patient that is affiliated with the Practice.
  \item The system shall allow Nurses to only view details of patients' treatments, tests or test results.
  \item The system shall allow General Practice Administrators to view, make or delete appointment details between patients and General Practitioners.
  \item The system shall allow General Practice Administrators to create statistical reports of the Practice's performance, such as the number of appointments per month, the number of test kits ordered, etc.
  \item The system shall allow Ambulance Service Administrators to log calls in real-time.
  \item The system shall allow the Ambulance Service Administrators to dispatch an ambulance that is closest to the region where the call was logged.
  \item The system shall allow the Ambulance Service Administrators to view patient records and details.
  \item The system shall allow the patient records to be updated by the Ambulance Service Administrator, to indicate that treatment of some kind has taken place or that the patient has been taken to A\&E.
  \item The system should allow Ambulance Service Crew to indicate when each call-out has been addressed.
\end{enumerate}




\subsection{Non-Functional Requirements}
\label{sect: requirements: non-functional}
The non-functional requirements can be divided into three categories: product requirements, organisational requirements, and external requirements.

\subsubsection*{Product Requirements}


\begin{enumerate}
  \item The system shall be able to handle all users that attempt to access, update or modify patient details, treatments given or diagnoses made.
  \item The system shall be able to process all identifiable data under the Data Protection Act,
        1998.
  \item The system shall be able to obtain real-time patient vitals in 0.1 seconds, load all operations within 1 second and process all requests within 2 seconds.
  \item The system shall be available between 24 hours a day, 7 days a week, except for maintenance breaks, which should be stated at least 7 days in advance.
  \item The system should have centralised logs that can maintain all services, instances and
        possible errors in a single location.
  \item The system shall be accessible via mobile devices and personal computers.
\end{enumerate}







\subsubsection*{Organisational Requirements}


\begin{enumerate}
  \item The system shall be down for maintenance, on the last working day of each calendar month, for 30 minutes.
  \item The system shall be evaluated for response times on the last working day of each calendar month.
  \item The system shall be able to prevent cross-scripting attacks.
  \item The system will not store unencrypted sensitive data.
  \item All patient data will be retained in system archives for up to 5 years.
\end{enumerate}



\subsubsection*{External Requirements}

\begin{enumerate}
  \item The system should be able to support an annual growth of 10 General Practices that can utilise this platform.
  \item The system should be able to support a 100\% growth in user concurrency, and still meet all defined functional and non-functional requirements.
\end{enumerate}




\subsection{Assumptions}
\label{sect: requirements: assumptions}

In order to architect the system, there were a number of assumptions that were made in relation to the scenario. These assumptions and their justifications, are listed as follows.



\begin{enumerate}

  \item \textbf{A Health Board Administrator is responsible for redirecting a referred patient to the nearest Hospital that offers their required service.}

        A Health Board could encompass multiple hospital; for example, the Fife Health Board could include Adamson Hospital (Cupar), Victoria Hospital (Kirkcaldy), and St Andrews Community Hospital. Thus, when a patient is \say{referred} to a local Health Board, it is assumed that the GP does not refer the patient to a specific hospital.

        It is assumed that a Health Board Administrator is responsible for selecting the specific hospital that the patient will be referred to, and as such, will pass the patient's details onto this specific hospital. This will ensure that the patient's details are communicated on a need-to-know basis, thereby reducing the chances of sensitive information being released to inappropriate individuals, improving the consideration of data security within the proposed architecture.

  \item \textbf{The Ambulance Service Administrator is able to dispatch ambulances that are nearest to the individual calling for assistance.}

        This indicates that the Ambulance Service Administrator retains a degree of autonomy when dispatching ambulances, in order to ensure that assistance is provided as quickly as possible. The architecture therefore does not consider this concern.

  \item \textbf{Different members within the GP, Health Board, and Ambulance Crew have different access permissions with respects to patient data.}

        Primarily, it is assumed that General Practitioners have access to all patient information. This includes the patient vitals (heart-rate, heart-rate variability, oxygen saturation, respiration rate, parasympathetic activity, etc), sensitive patient information (date of birth, gender, height, weight, etc), and patient diagnosis and treatment information.

        The data which users are authorised to view is assumed to be in concordance with the access permissions listed in the scenario in Appendix \ref{appendix: scenario}, with unreferenced data permissions being extrapolated appropriately where required.

  \item \textbf{All medical personnel records are registered in the National Medical Professionals' Database.}

        It is assumed that the credentials utilised to authenticate users are obtained from a centralised database that contains the information of all registered medical professionals within the country.

        This assumption allows for the system architecture to ensure that only authorised medical personnel are allowed to access patient records and that any externally created user accounts, such as system administrators, are automatically unauthorised to view sensitive information.

\end{enumerate}





\subsection{Functionality}
\label{sect: process: functionality}

In order to design an architectural framework that is able to accurately execute the required functionality of the system, it was important to conceptualise the expected logical flow of events for different parts of the system.

Figures \ref{fig: appointments flow diagram} and \ref{fig: ambulance flow diagram} display an overview of the event flows which are associated with a patient making an appointment and the system's ambulance-related capabilities respectively. These flow diagrams were used to influence the functionality which the proposed architecture was designed to support.


\vspace{4mm}
\begin{figure} [H]
  \centering
  \includegraphics[scale = 0.64]{appt-flow.png}
  \caption{The flow of events relating how appointments are handled by the system.}
  \label{fig: appointments flow diagram}
\end{figure}



\vspace{4mm}
\begin{figure} [H]
  \centering
  \includegraphics[scale = 0.64]{ambulance-flow.png}
  \caption{The flow of events relating to the system's handling of ambulance calls.}
  \label{fig: ambulance flow diagram}
\end{figure}


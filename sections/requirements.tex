% A list of functional and non-functional requirements of the system along with their priorities, and any design or business constraints

During the initial design process (see \S\ref{sect: process}), a set of requirements were derived from the system specification listed in Appendix \ref{appendix: scenario}. The functional requirements from this set are displayed in Section \ref{sect: requirements: functional} and the non-functional requirements are displayed in Section \ref{sect: requirements: non-functional}.



\subsection{Functional Requirements}
\label{sect: requirements: functional}

\begin{enumerate}
  \item The system shall be able to allow patients and General Practitioners to manage, create and facilitate digital appointments.
  \item The system shall be able to securely store sensitive patient data.
  \item The system shall be able to authenticate patients that interact with the application, by verifying their details against data obtained from the General Practitioners' records.
  \item The system shall be able to authenticate General Practitioners, General Practice Administrators, Nurses, Ambulance Service Administrators and Health Board Administrators that interact with the application, by verifying their details against data obtained from the National Medical Professionals Database.
  \item The system shall allow the patient records to be viewed and updated by the General Practitioner.
  \item The system shall allow the General Practitioner to refer the patient to the Local Health Board if the required service is available locally, or to the National Health Board if the required system is not available locally.
  \item The system shall allow the patient records to be updated by the Health Board Administrator, when the patient goes to A\&E in a Hospital, to indicate that treatment of some kind has taken place.
  \item The system shall allow Hospital doctors to view patient details and vitals, and add entries on diagnoses made and treatment given.
  \item The system shall notify the General Practitioners when the Ambulance Service Administrator or Health Board Administrator has updated the patient details.
  \item The system shall allow patients to view their vital data, and any changes made as a result of their digital appointment.
  \item The system shall allow General Practitioners to order tests for a specific patient that is affiliated with the Practice.
  \item The system shall allow Nurses to only view details of patients' treatments, tests or test results.
  \item The system shall allow General Practice Administrators to view, make or delete appointment details between patients and General Practitioners.
  \item The system shall allow General Practice Administrators to create statistical reports of the Practice's performance, such as the number of appointments per month, the number of test kits ordered, etc.
  \item The system shall allow Ambulance Service Administrators to log calls in real-time.
  \item The system shall allow the Ambulance Service Administrators to dispatch an ambulance that is closest to the region where the call was logged.
  \item The system shall allow the Ambulance Service Administrators to view patient records and details.
  \item The system shall allow the patient records to be updated by the Ambulance Service Administrator, to indicate that treatment of some kind has taken place or that the patient has been taken to A\&E.
  \item The system should allow Ambulance Service Crew to indicate when each call-out has been addressed.
\end{enumerate}




\subsection{Non-Functional Requirements}
\label{sect: requirements: non-functional}
The non-functional requirements can be divided into three categories: product requirements, organisational requirements, and external requirements.

\subsubsection*{Product Requirements}


\begin{enumerate}
  \item The system shall be able to handle all users that attempt to access, update or modify patient details, treatments given or diagnoses made.
  \item The system shall be able to process all identifiable data under the Data Protection Act,
        1998.
  \item The system shall be able to obtain real-time patient vitals in 0.1 seconds, load all operations within 1 second and process all requests within 2 seconds.
  \item The system shall be available between 24 hours a day, 7 days a week, except for maintenance breaks, which should be stated at least 7 days in advance.
  \item The system should have centralised logs that can maintain all services, instances and
        possible errors in a single location.
  \item The system shall be accessible via mobile devices and personal computers.
\end{enumerate}







\subsubsection*{Organisational Requirements}


\begin{enumerate}
  \item The system shall be down for maintenance, on the last working day of each calendar month, for 30 minutes.
  \item The system shall be evaluated for response times on the last working day of each calendar month.
  \item The system shall be able to prevent cross-scripting attacks.
  \item The system will not store unencrypted sensitive data.
  \item All patient data will be retained in system archives for up to 5 years.
\end{enumerate}



\subsubsection*{External Requirements}

\begin{enumerate}
  \item The system should be able to support an annual growth of 10 General Practices that can utilise this platform.
  \item The system should be able to support a 100\% growth in user concurrency, and still meet all defined functional and non-functional requirements.
\end{enumerate}




\subsection{Assumptions}
\label{sect: requirements: assumptions}

In order to architect the system, there were a number of assumptions that were made in relation to the scenario. These assumptions and their justifications, are listed as follows.



\begin{enumerate}

  \item \textbf{A Health Board Administrator is responsible for redirecting a referred patient to the nearest Hospital that offers their required service.}
        For example, a student based in St. Andrews would have their patient details and records stored within the Fife Health Board. This is the Health Board that the local General Practice (example- Blackfriars or Pipeland) would refer the patient to.

        A Health Board could encompass multiple Hospitals - for example, the Fife Health Board could include Adamson Hospital (Cupar), Victoria Hospital (Kirkcaldy), St Andrews Community Hospital, etc. Thus, when a patient is “referred” to a local Health Board, we assume that the General Practice does not refer the patient to a specific Hospital.

        Hence, we assume that a Health Board Administrator is responsible for selecting the specific Hospital that the patient will be referred to, and as such, will pass the patient's details (vitals, treatment records, diagnoses, personal information, etc) onto this specific hospital. This will ensure that the patient's details are communicated on a need-to-know basis, thereby reducing the chances of sensitive information being released to inappropriate individuals.

  \item \textbf{The Ambulance Service Administrator is able to dispatch ambulances that are nearest to the individual calling for assistance.}
        This indicates that the Ambulance Service Administrator retains a degree of autonomy when dispatching ambulances, in order to ensure that assistance is provided as quickly as possible.

        An example of a scenario where this would be useful would be a call requiring assistance to be sent to St Andrews. If there are two ambulances in Kirkcaldy and Cupar, we would want the Ambulance Service Administrator to be able to send the ambulance in Cupar, as it is closer.

        Hence, we do not expect the system to be automatically dispatching ambulances, as the calls come in.
        However, we do expect the system to be able to provide a recommendation of which ambulance is closest (computed using third-party navigation software), which would help the Ambulance Service Administrator dispatch assistance.


  \item \textbf{Different members of the General Practice/Health Board/Ambulance Crew have different access permissions, regarding patient data.}
        Primarily, we assume that General Practitioners have access to all patient information. This includes the patient vitals (heart-rate, heart-rate variability, oxygen saturation, respiration rate, parasympathetic activity, etc), sensitive patient information (date of birth, gender, height, weight, etc), patient diagnoses and treatment.

        We assume that Nurses only have access to patient diagnoses and treatment.
        We assume that General Practice Administrators have access to no patient data. We assume that these individuals can only view anonymised appointment data.

        We assume that Health Board Administrators have access to no patient data, except the information that the General Practitioner has sent over, regarding the services required by the patient.

        We assume that the Hospital doctors have read-only access to patients' vitals and sensitive data, but can update patient diagnoses and treatment.

        We assume that the Ambulance Service Administrators have no access to patient data, except the data communicated by the patient during the ambulance call.

        We assume that the Ambulance Service Crew have read-only access to patient vitals and sensitive patient information, but can update patient diagnoses and treatment.


  \item \textbf{Patients are required to make a digital appointment when being seen by a Nurse/General Practitioner in person.}
        We assume that the patient must make an appointment by phoning the General Practice, in order to be seen by a Nurse or General Practitioner in real life. This allows us to model real-world behaviour as closely as possible, as we would expect that General Practices require an appointment to be made, in advance. This holds true whether it is a digital appointment or an in-person appointment.
        We also assume that the steps following an in-person appointment are the same as those that follow a digital appointment. Specifically, the patient's vitals are taken and added to the database, treatment is offered/the patient is referred, and test kits are ordered.


  \item \textbf{All medical personnel records are registered in a comprehensive database - the National Medical Professionals' Database.}
        We assume that the credentials utilised to authenticate users are obtained from a centralised database that contains the information of all registered medical professionals within the country. This allows the system architecture to ensure that only authorised medical personnel are allowed to access patient records and that any externally created user accounts (such as system administrators) are automatically unauthorised to view sensitive information.

\end{enumerate}





\subsection{Functionality}
\label{sect: process: functionality}
In order to design an architectural framework that is able to accurately execute the required functionality of the system, it was imperative to conceptualise the expected logical flow of events for different parts of the system.

Figure \ref{fig: appointments flow diagram} is a brief overview of the chain of events that are associated with a patient making an appointment. As seen above, a patient attends a digital appointment that enables the General Practitioner to collect their vitals, using the Binah.AI technology, which is then entered into the General Practice's database. Once this is done, the General Practitioner can either treat the patient, or refer the patient to the local Health Board, if the patient requires a form of assistance that is not provided by the General Practice.

If the patient is treated by the General Practice, the patient records are updated with the diagnoses made and the treatments administered.

If they are referred to the local Health Board, the patient is referred to the closest Hospital that can provide the appropriate treatment. Once this has been done, a Hospital doctor treats the patient, and updates the patient records with the diagnoses made and the treatments administered. The General Practitioner of the patient is then notified.

From this brief diagram of the base case, one is able to extract a number of key inferences about the way the data should behave, the access permissions of different individuals regarding patient data, and the inter-component behaviours.



\vspace{4mm}
\begin{figure} [H]
  \centering
  \includegraphics[scale = 0.5]{appt-flow.png}
  \caption{The flow of events relating how appointments are handled by the system.}
  \label{fig: appointments flow diagram}
\end{figure}



Following this, Figure \ref{fig: ambulance flow diagram} extends the functionality of the system. This figure illustrates the chain of events that follow, once a call for emergency assistance has been placed.

Once a patient has identified that they require emergency assistance, they can either phone the Ambulance Services or go to A\&E (in their local Hospital).

If the Ambulance Services are phoned, the nearest Ambulance Service Crew is dispatched, who can either treat the patient or take them to A\&E. If they are treated by the Crew, the patient records are updated with the newly administered treatment and the diagnoses made.

If the patient chooses to go to A\&E themselves, or the crew takes them to A\&E, the patient is treated by a doctor at the Hospital, who, like the Ambulance Service Crew, can update their diagnoses and treatment records.
Once the patient records have been updated, either by the Ambulance Service Crew or the doctor, the General Practice is notified.

Similar to Figure \ref{fig: appointments flow diagram}, one can note a number of entities that need to access the same services or perform similar functionalities, thereby streamlining the architectural expectations for this system, which makes it simpler to design the framework.

\vspace{4mm}
\begin{figure} [H]
  \centering
  \includegraphics[scale = 0.5]{ambulance-flow.png}
  \caption{The flow of events relating to the system's handling of ambulance calls.}
  \label{fig: ambulance flow diagram}
\end{figure}


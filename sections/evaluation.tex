\subsection{Current Considerations}
\label{sect: evaluation: current considerations}

\subsubsection*{Authentication Performance Bottleneck}
- SAM
- All devices must communicate with auth service
- Can be imrpoved by replicating and distributing replicas, but this adds significant amount of complexity


\subsubsection*{Architect Biases}
- IRA
- did a lot of research on styles
- natural biases
- influence from past project \& current trends
- SOA is held highly
- monolithic layered are shunned (good for critical system)


\subsubsection*{Data Loss in Event-Driven Styles}
The use of event-driven styles within the architecture includes the pub-sub messaging systems employed in the GP Notification Service (see \S\ref{sect: architecture: gp notification svc}) and the Ambulance Service (see \S\ref{sect: architecture: ambulance svc}), as well as the multilevel priority queue used in the Patient Information Service (see \S\ref{sect: architecture: patient information svc}). Each of these applications of event-driven architecture comes with the risk of data being lost, as the intermediate stores for the respective \say{events} are memory-based, meaning that data will be lost upon service failure if the correct precautions are not taken.

The proposed architecture takes step to mitigate data loss within the GP Notification Service by backing-up notifications which are buffered within the pub-sub messaging system to a resilient datastore (see \S\ref{sect: architecture: gp notification svc}); the Ambulance Service's pub-sub system also has similar functionality.

However, the multilevel priority queue within the Patient Information Service does not currently have any architectural support for providing fault-tolerance with respect to write requests which arrive in the queue. Improvements to the architecture could include the use of a simple append-only log-file to reduce the amount of data lost in a fault, and more complex fault-tolerance mechanics could be applied to mitigate almost all data loss; however, the overhead introduced by such a scheme could be counterintuitive as much patient data (e.g. blood pressure readings) could likely be regenerated fairly easily.



\subsubsection*{BFF \& Gateway Patterns}
- SAM
- using both rather than just one might be considered unusual \& come with unnecessary overhead
- reuse functionality through gateway (e.g. security filters)
- some can't be (e.g. req coalescing)
- Combining both is beneficial



\subsubsection*{Patient Information Service Bloating}
- SAM
\todo{rename sect}

- one svc vs many services
- choice driven by component principles

\subsubsection*{Pub-Sub Message Loss}
- SAM
- potential issue
- relies on implementation to get correct (developer vigilance)




\subsubsection*{Size}
- IRA

- Splitting into teams could be Tricky
- Overarching architects or not? (trade-offs)






\subsubsection*{Data Security}
One key consideration for the target system is that of data security, as patient health data can be considered to be very sensitive information \cite{EUHealthDataProtection}. In evaluating the proposed architecture, it can be seen that the architecture supports the implementation of data security in the following ways.

\begin{itemize}
  \item All sensitive data is stored in a single datastore within the Patient Information Service (see \S\ref{sect: architecture: patient information svc}). This means that the introduction and management of security policies and measures is greatly facilitated during the system's implementation and lifecycle.
  \item The integration of the OAuth 2 framework within the architecture facilitates the secure authentication of users using a modern protocol (see \S\ref{sect: architecture: authentication}).
\end{itemize}




\subsection{Lifecycle Considerations}
\label{sect: evaluation: lifecycle considerations}

\subsubsection*{System Evolution}
- IRA
- Arch is flexible \& maintainable w.r.t. future changes
- Application of component principles helps with this
- Documentation


\subsubsection*{Architectural Drift \& Erosion}
- IRA

- Documentation

- From lack of dev vigilance:
- stick with arch
- uphold boundaries
- write docs
- remove cruft (cite Martin Fowler)
- team
- some hoisting has been applied (e.g. service separation), but still left to developers
- SOA is not \textbf{that} important in an arch, but instead we should focus on boundaries (see Clean Architectrue book ch 27 \cite{cleanArch-Services})

\subsubsection*{ATAM}


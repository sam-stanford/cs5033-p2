
You are required to develop an architecture for a patient digital appointments and
management system.

This system should act as a central point for creating and delivering patient
appointments as well as storing patient data. Digital appointments should be created
via this system, allowing a GP to ascertain and monitor vital patient data. The patient's
own data should also be stored in the system allowing for a full 360 degree view of
patient information. This system is being built in order to alleviate pressure on the health
service.

Each person resident in Scotland is registered with a General Practitioner (GP)
practice. The health care service will maintain a record of each person's name, a unique
health care id, date of birth, contact details, information on next of kin, the GP practice
they are registered with and medical history. Patient records contain highly sensitive
information. A patient may be seen via digital appointments or at the practice by a nurse
(for minor injuries or tests) or a GP. A patient can only access the GP practice if they
are referred to after a digital appointment, except in case of emergencies when they
can either call an ambulance or go directly to the accident and emergency (A\&E)
department of a hospital. An ambulance crew will either treat a patient on site or take
them to a hospital. A GP will either treat a patient themselves or refer the patient to a
hospital run by the local health board, except when the required service is not available
locally. Each appointment or treatment will result in an entry in the person's medical
history, which should be accessible across different health boards and services. When
a patient has been treated at a hospital or by an ambulance crew, a notification should
be sent to their GP to flag up the new entry.

The GP should be able to generate digital appointments for patients. These
appointments should allow the GP (via the patient's personal device) to obtain vital
information about the patient – namely their blood pressure, heart rate, heart rate
variability, oxygen saturation and respiration rate. This data should be generated in
real-time and stored securely in the patient information system.
The system should provide different functionalities depending on the category of user.
A patient must be able to use the service to make appointments at their GP practice.
They should also be able to see all the vital data generated by all their digital
appointments in the system. GPs should be able to access the full record of any patient
without delay, initiate appointments and add entries to the patient record. They should
also be able to order one or more tests for the patient within the practice or refer the
patient to a hospital, either within or outside the local health board. Nurses can see a
limited part of the patient record and add entries relating to treatment, tests or test
results. Practice administrators can make or cancel appointments for patients and
produce statistical reports on the performance of the practice without accessing details
of any patient.


An ambulance service administrator must be able to log calls to the service and
dispatch ambulances to patients who require them. This aspect of the service is highly
time sensitive. They can also update patients' records according to the service
delivered. Ambulance crews indicate when each callout has been dealt with.
Hospital doctors can view all the details of any patient they see and add entries on
diagnoses made and treatment given.

A well-designed and implemented system should also support the following features:
\begin{itemize}
  \item An intuitive UI appropriate to the user category,
  \item Access authentication for different categories of users and restriction of
        available information and functionality accordingly,
  \item Support for multiple types of devices,
  \item Concurrent access,
  \item Support for several different views and analyses over the data,
  \item Validation of input data where applicable,
  \item Generation of digital appointments and using patients' devices functionalities to
        extract the necessary patient data, and
  \item Deal with potential uncertainties when patients' devices cannot extract this data.
\end{itemize}

\subsection{Resources}
\label{sect: process: resources}

The primary literature resources used to influence the design, evaluation, and implementation plan of the chosen architecture were Robert C. Martin's Clean Architecture \cite{cleanArch-WholeBook}, and Mark Richards and Neal Ford's Fundamentals of Software Architecture \cite{fundamentalsOfSWA-WholeBook}.

Clean Architecture provides information on architecture from a practical standpoint, including fundamental principles to follow when architecting components and code, as well as a critical analysis of modern software-architecture conventions. The principles relating to the architecture of code which explored in the book are the SOLID design principles, which are summarised as follows.

\todo{summarise}

\begin{itemize}
  \item \textbf{Single-Responsibility Principle (SRP)}:
  \item \textbf{Open-Closed Principle (OCP)}:
  \item \textbf{Liskov Substitution Principle (LSP)}:
  \item \textbf{Interface Segregation Principle (ISP)}:
  \item \textbf{Dependence Inversion Principle (DIP)}:
\end{itemize}

The principles relating to the architecture of components which are explored Clean Architecture are divided into two groups: those which deal with component cohesion, and those which deal with component coupling. The principles relating to component cohesion are as follows.

\begin{itemize}
  \item \textbf{Reuse/Release Equivalence Principle (REP)}:
  \item \textbf{Common Closure Principle (CCP)}:
  \item \textbf{Common Reuse Principle (CRP)}
\end{itemize}

The principles relating to component coupling in Clean Architecture are as follows.

\begin{itemize}
  \item \textbf{Acyclic Dependencies Principle (ADP)}:
  \item \textbf{Stable Dependencies Principle (SDP)}:
  \item \textbf{Stable Abstractions Principle (SAP)}:
\end{itemize}

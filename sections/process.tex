\subsection{Architecting Phases}
\label{sect: process: phases}

The process that was used to derive this architecture was consisted of two primary phases, were the Planning Phase and Design Phase.

\subsubsection*{Planning Phase}
The first phase to take place was the Planning Phase, which focussed on determining the requirements which the architecture should support and the characteristics which the architecture should display. The determined requirements are listed in Section \ref{sect: requirements}, and the characteristics which were highlighted during discussion are as follows.

\begin{enumerate}
  \item \textbf{Security}: One of the most important characteristics for the architecture to support is the security of the system. Health data is sensitive for a number of reasons, with its leakage potentially putting the individuals whose health information was infringed in vulnerable position \cite{EUHealthDataProtection}. Both the NHS' documentation regarding data security \cite{nhsDataSecurity} and other resources, such as the HIPAA guidelines \cite{HIPAA-Guidelines}\cite{HIPAA-Privacy-WholeBook}, were consulted during the architecture's design in order to cater for this concern.

  \item \textbf{Robustness}: The robustness of the system is a critical consideration, as healthcare systems must be operations for 24 hours a day, 7 days a week. This is because medical emergencies can happen at any time, and it is essential for relating systems to be able to handle requests, such as extracting patient records or viewing recent treatments, at any given time. As such, it was determined that the system should continue to function, despite any faults within its constituent subsystems, leading to a component-based architecture being used.

  \item \textbf{Performance}: The final essential consideration was the performance of the system. As medical professionals require the reading and writing of patient data in real-time, it is essential that the system is able to respond to requests within a specific frame of time. Hence, considerations such as latency and channel capacity were considered essential during the development of the proposed architecture.
\end{enumerate}





\subsubsection*{Design Phase}
The design of the architecture took place iteratively, with initial architectural designs being completed by different group members, and later being combined into a final architecture which was repeatedly refined.

All designs are centred around the key stakeholders and actors of the system, including GP staff, Ambulance Service staff, Hospital staff, and patients, as well as their associated tasks and behaviours. For example, architectures would consider the GPs and how their tasks of making and attending digital appointments, order test kits, etc. could be supported by the architecture. Considering the required functionality while designing the architecture enabled the discovery of the most effective architectural frameworks for the constituent subsystems.

Throughout the design process, several ambiguities within the scenario were discovered, leading to a set of assumptions to be made for the final architectural design; these assumptions are presented in Section \ref{sect: requirements: assumptions}. Alongside the consideration of actors and requirements, fundamental architectural styles and patterns that could be utilised within the system were discussed. Discussed styles and patterns include the publish-subscribe pattern for the Ambulance Service Crew, and the event-driven style for the General Practitioner.

Figure \ref{fig: whiteboard sketch} displays the final architecture which was determined during the Design Phase, where it can be seen that the stakeholders of the system are illustrated, including as the GP staff, Hospital staff, patients, and Ambulance Service staff, along with the chosen services that are represented in circles. The different entities that interact with these activities are represented through arrows, which makes it easier to conceptualise the communication required between actors and services.

The bottom right of the diagram in Figure \ref{fig: whiteboard sketch} displays a proposal for the architecture of the Patient Information Service. It can be seen that the proposed architecture is similar to that of the final architecture of the service, which is discussed in Section \ref{sect: architecture: patient information svc}.

\vspace{4mm}
\begin{figure} [H]
  \centering
  \includegraphics[scale = 0.4]{Whiteboard.png}
  \caption{Initial whiteboard sketch of the proposed architecture.}
  \label{fig: whiteboard sketch}
\end{figure}














\subsection{Resources}
\label{sect: process: resources}

The primary literature resources used to influence the design, evaluation, and implementation plan of the chosen architecture were Robert C. Martin's Clean Architecture \cite{cleanArch-WholeBook}, and Mark Richards and Neal Ford's Fundamentals of Software Architecture \cite{fundamentalsOfSWA-WholeBook}.

Fundamentals of Software Architecture was used to influence the choice and application of different architectural styles, patterns, and features. Clean Architecture provides information on architecture from a practical standpoint, including fundamental principles to follow when architecting a system, including a set of component principles, which influenced the design of the proposed architecture. These component principles are divided into two groups: those which deal with component cohesion, and those which deal with component coupling. The principles relating to component cohesion are as follows.

\begin{itemize}
  \item \textbf{Reuse/Release Equivalence Principle (REP)}: Component constituents which are released together should be reused together and vice versa.
  \item \textbf{Common Closure Principle (CCP)}: Component constituents which change for the same reason and at the same time should be placed in the same component.
  \item \textbf{Common Reuse Principle (CRP)}: Components should not contain excess constituents in a way which would force users to depend on things they do not need.
\end{itemize}

The principles relating to component coupling in Clean Architecture are as follows.

\begin{itemize}
  \item \textbf{Acyclic Dependencies Principle (ADP)}: The component dependency graph should not contain any cycles.
  \item \textbf{Stable Dependencies Principle (SDP)}: Dependencies between components should point in the direction of stability, leading to most dependencies pointing in the direction of the system's business rules.
  \item \textbf{Stable Abstractions Principle (SAP)}: Components should be as abstract as they are stable. For example, a stable component (e.g. a business rule implementation) should be implemented as interfaces and abstract classes, while a volatile component (e.g. a web interface) should try to avoid such abstractions.
\end{itemize}



\subsection{Technologies}
\label{sect: process: technologies}
Throughout the design process, the following technologies were used.

\begin{itemize}
  \item \textbf{Microsoft Teams} \cite{msTeams}: Used for group text and video communication.
  \item \textbf{Google Drive} \cite{googleDrive}: Used to store and share images, progress reports, meeting notes, and other documents.
  \item \textbf{LaTeX} \cite{latex}: Used for document formatting.
  \item \textbf{GitHub} \cite{github}: Used to store LaTeX code and enable asynchronous work on the report by different team members.
  \item \textbf{Lucidchart} \cite{lucidChart}: Used to create architecture diagrams.
  \item \textbf{diagrams.net} \cite{diagramsDotNet}: Used to create system functionality flow charts.
\end{itemize}
